\documentclass[preprint,12pt]{elsarticle}

\usepackage{hyperref}
\usepackage{graphicx}
\usepackage{subcaption}
\usepackage{amssymb}
\usepackage{amsmath}
\usepackage{multirow}
\usepackage{booktabs}
\usepackage[utf8]{inputenc}
\usepackage{cleveref}
\usepackage[section]{placeins}

% For the TODOs
\usepackage{xcolor}
\usepackage{xargs}
\usepackage[colorinlistoftodos,textsize=footnotesize]{todonotes}
\newcommand{\todoin}{\todo[inline]}
% from here: https://tex.stackexchange.com/questions/9796/how-to-add-todo-notes
\newcommandx{\unsure}[2][1=]{\todo[linecolor=red,backgroundcolor=red!25,bordercolor=red,#1]{#2}}
\newcommandx{\change}[2][1=]{\todo[linecolor=blue,backgroundcolor=blue!25,bordercolor=blue,#1]{#2}}
\newcommandx{\info}[2][1=]{\todo[linecolor=OliveGreen,backgroundcolor=OliveGreen!25,bordercolor=OliveGreen,#1]{#2}}

%Boldtype for greek symbols
\newcommand{\teng}[1]{\ensuremath{\boldsymbol{#1}}}
\newcommand{\ten}[1]{\ensuremath{\mathbf{#1}}}

\usepackage{lineno}


\journal{}

\begin{document}

\begin{frontmatter}

  \title{Fluid structure interaction with ETVF}
  \author[IITB]{A Dinesh\corref{cor1}}
  \ead{adepu.dinesh.a@gmail.com}
  \author[IITB]{Prabhu Ramachandran}
  \ead{prabhu@aero.iitb.ac.in}
  \address[IITB]{Department of Aerospace Engineering, Indian Institute of
    Technology Bombay, Powai, Mumbai 400076}

\cortext[cor1]{Corresponding author}

\begin{abstract}
\end{abstract}

\begin{keyword}
%% keywords here, in the form: keyword \sep keyword
{XXX}, {XXX}, {XXX}

%% MSC codes here, in the form: \MSC code \sep code
%% or \MSC[2008] code \sep code (2000 is the default)

\end{keyword}

\end{frontmatter}

% \linenumbers



\section{Introduction}
\label{sec:intro}

Xu has introduced the particle shifting and adjusted the particles and applied
it to fluid problems. TVF is a new variant which unifies the formulation and
applied it to fluid problems. This TVF has been generalized and applied to free
surface problems as well as elastic solid problems.



\section{Governing equations}

continuity equation
\begin{equation}
  \label{eq:ce}
  \frac{d \rho}{d t} = - \rho \; \frac{\partial u_i}{\partial x_i},
\end{equation}
and by the momentum equation
\begin{equation}
  \label{eq:me}
  \frac{d u_i}{d t} = \frac{1}{\rho} \; \frac{\partial \sigma_{ij}}{\partial x_j}
  + g_i,
\end{equation}
with $\frac{{d} }{d t}$, $\rho$, $t$, $u_i$, $x_i$, $\sigma_{ij}$ and $g_{i}$
denoting material derivative, density, time, velocity vector, position vector,
stress tensor and gravity, respectively.


The stress tensor is split into isotropic and deviatoric parts,
\begin{equation}
  \label{eq:stress_tensor_decomposition}
  \sigma_{ij} = - p \; \delta_{ij} + \sigma'_{ij},
\end{equation}
where $p$ and $\delta_{ij}$ denote pressure and Kronecker delta function, while
$\sigma'_{ij}$ is the deviatoric stress.

The rate of change of deviatoric stress $\frac{d \sigma'_{ij}}{dt}$ can be
computed from the strain rate $\dot{\epsilon}_{ij}$ and rotation tensor
$\dot{\Omega}_{ij}$ according to Jaumann's formulation for Hooke's stress law
as
\begin{equation}
  \label{eq:jaumann-stress-rate}
  \frac{d \sigma'_{ij}}{dt} = 2G (\dot{\epsilon}_{ij} - \frac{1}{3}
  \dot{\epsilon}_{kk} \delta_{ij}) + \sigma^{'}_{ik}  \Omega_{jk} +
  \Omega_{ik} \sigma^{'}_{kj}.
\end{equation}
Here $G$ is the shear modulus, while $\dot{\epsilon}_{ij}$ is
\begin{equation}
  \label{eq:strain-tensor}
  \dot{\epsilon}_{ij} = \frac{1}{2} \bigg(\frac{\partial u_i}{\partial x_j} +
  \frac{\partial u_j}{\partial x_i} \bigg),
\end{equation}
and $\Omega_{ij}$ is
\begin{equation}
  \label{eq:rotational-tensor}
  \Omega_{ij} = \frac{1}{2} \bigg(\frac{\partial u_i}{\partial x_j} -
  \frac{\partial u_j}{\partial x_i} \bigg).
\end{equation}

For a weakly-compressible or incompressible fluid the deviatoric stress
vanishes and a viscous force is added to the stress tensor:
\begin{equation}
  \label{eq:fluid-stress-decomposition}
  \sigma_{ij} = - p \delta_{ij} + 2 \eta \frac{\partial u_i}{\partial x_j}
\end{equation}
where $\eta$ is the kinematic viscosity of the fluid.

In both fluid and solid modelling the pressure is computed using an
isothermal equation of state, given as,
\begin{equation}
  \label{eq:pressure-equation}
  p = K \bigg(\frac{\rho}{\rho_{0}} - 1 \bigg),
\end{equation}
where $K = \rho_{0} c_0^2$ is the bulk modulus. Here, the constants $c_0$ and
$\rho_0$ are the reference speed of sound and density, respectively. For solids,
$c_0$ is computed as $\sqrt{\frac{E}{3 (1 - 2 \nu)\rho_{0}}}$, $\nu$ is the
Poisson ratio.


\section{Numerical method}

Write about Material derivative

\begin{equation}
  \label{eq:modified-material-derivative}
  \frac{\tilde{d} }{d t} = \frac{\partial }{\partial t} +
  \tilde{u}_j \frac{\partial }{\partial x_j}.
\end{equation}

\begin{equation}
  \label{eq:ce-tvf}
  \frac{\tilde{d} \rho}{d t} =
  - \rho \frac{\partial \tilde{u}_j}{\partial x_j} +
  \frac{\partial (\rho (\tilde{u}_j - u_j))}{\partial x_j}.
\end{equation}

\begin{equation}
  \label{eq:mom-tvf}
  \frac{\tilde{d} u_i}{d t} =
  \frac{\partial}{\partial x_j} (u_i (\tilde{u}_j - u_j))
  - u_i \frac{\partial}{\partial x_j} (\tilde{u}_j - u_j)
  + g_i
  +\frac{1}{\rho} \frac{\partial \sigma_{ij}}{\partial x_j}.
\end{equation}


\begin{equation}
  \label{eq:edac-p-evolve}
  \frac{\tilde{d} p}{d t} =
  (p -\rho c_s^2)
    \text{div}(\ten{u})
  - p \; \text{div}(\tilde{\ten{u}})
    + \text{div}(p (\tilde{\ten{u}} - \ten{u}))
    + \nu_{edac}  \nabla^2 p.
\end{equation}
%
The value of $\nu_{edac}$ is,
\begin{equation}
  \label{eq:nu-edac}
  \nu_{edac} = \frac{\alpha_{\textrm{edac}} h c_s}{8},
\end{equation}


\subsection{SPH discretization}

In the current work, both fluid and solid modelling uses the same continuity
and pressure evolution equation. The SPH discretization of the continuity
equation~\eqref{eq:ce-tvf} and the pressure evolution
equation~\eqref{eq:edac-p-evolve} respectively are,
\begin{equation}
  \label{eq:sph-discretization-continuity}
  \frac{\tilde{d}\rho_a}{dt} = \sum_{b} \; \frac{m_b}{\rho_{b}} \; (
  \rho_{a} \; \tilde{\ten{u}}_{ab} \; + \;
  (\rho \; (\tilde{\ten{u}} \; - \;
  \ten{u}))_{ab}) \; \cdot \nabla_{a} W_{ab},
\end{equation}

\begin{multline}
  \label{eq:sph-discretization-edac}
  \frac{\tilde{d}p_a}{dt} = \sum_{b} \; \frac{m_b}{\rho_{b}} \; \bigg(
  (p_{a} - \rho_{a} c_{s}^2) \; \ten{u}_{ab} \; + \;
  p_{a} \; \tilde{\ten{u}}_{ab} \; - \;
  (p \; (\tilde{\ten{u}} - \ten{u}))_{ab} \; + \; \\
  4 \; \nu_{edac}
  \frac{p_a - p_b}{(\rho_a + \rho_b) (r^2_{ab} + 0.01 h_{ab}^{2})} \ten{r}_{ab}
  \bigg) \; \cdot \nabla_{a} W_{ab}.
\end{multline}
%
Similarly, the discretized momentum equation for fluids is written as,
\begin{multline}
  \label{eq:sph-momentum-fluid}
  \frac{\tilde{d}\ten{u}_{a}}{dt} = - \sum_{b} m_b \bigg[
  \bigg(\frac{p_a}{\rho_a^2} + \frac{p_b}{\rho_b^2}\bigg) \ten{I} -
  \bigg(\frac{\ten{A}_a}{\rho_a^2} + \frac{\ten{A}_b}{\rho_b^2} + \Pi_{ab}
  \ten{I} \bigg) \bigg]
  \cdot \nabla_{a} W_{ab} \\
  + \ten{u}_{a} \sum_{b} \frac{m_b}{\rho_{b}} \; \tilde{\ten{u}}_{ab} \cdot
  \nabla_{a} W_{ab} + \sum_{b} m_b \frac{4 \eta \nabla W_{ab}\cdot
    \ten{r}_{ab}}{(\rho_a + \rho_b) (r_{ab}^2 + 0.01 h_{ab}^2)} \ten{u}_{ab} +
  \ten{g}_{a},
\end{multline}
where $\ten{A}_a = \rho_a \ten{u}_a \otimes (\ten{\tilde{u}}_a - \ten{u}_a)$,
$\ten{I}$ is the identity matrix, $\eta$ is the kinematic viscosity of the
fluid and \citet{morris1997modeling} formulation is used to discretize the
viscosity term. $\Pi_{ab}$ is the artificial
viscosity~\cite{monaghan-review:2005} to maintain the stability of the
numerical scheme. It is given as,
\begin{align}
  \label{eq:mom-av}
  \Pi_{ab} =
  \begin{cases}
\frac{-\alpha h_{ab} \bar{c}_{ab} \phi_{ab}}{\bar{\rho}_{ab}}
  & \ten{u}_{ab}\cdot \ten{r}_{ab} < 0, \\
  0 & \ten{u}_{ab}\cdot \ten{r}_{ab} \ge 0,
\end{cases}
\end{align}
where,
%
\begin{equation}
  \label{eq:av-phiij}
  \phi_{ab} = \frac{\ten{u}_{ab} \cdot \ten{r}_{ab}}{r^2_{ab} + 0.01 h^2_{ab}},
\end{equation}
%
where $\ten{r}_{ab} = \ten{r}_a - \ten{r}_b$, $\ten{u}_{ab} = \ten{u}_a -
\ten{u}_b$, $h_{ab} = (h_a + h_b)/2$, $\bar{\rho}_{ab} = (\rho_a + \rho_b)/2$,
$\bar{c}_{ab} = (c_a + c_b) / 2$, and $\alpha$ is the artificial
viscosity parameter.

%
For solid mechanics the momentum equation is written as,
\begin{equation}
  \label{eq:sph-momentum-solid}
  \frac{\tilde{d}\ten{u}_{a}}{dt} = - \sum_{b} m_b \bigg[
  \bigg(\frac{p_a}{\rho_a^2} + \frac{p_b}{\rho_b^2}\bigg) \ten{I} -
  \bigg(\frac{\teng{\sigma}^{'}_{a}}{\rho_a^2} +
  \frac{\teng{\sigma}^{'}_{b}}{\rho_b^2} + \Pi_{ab} \ten{I} \bigg) \bigg]  \cdot \nabla_{a} W_{ab} +
  \ten{g}_{a},
\end{equation}
we have not considered the correction stress term $\ten{A}$ in momentum
equation of solid mechanics as it has a negligible effect.

In addition to these three equations, the Jaumann stress rate equation is also
solved. In the current work we use the momentum velocity $\ten{u}$ rather than
$\tilde{\ten{u}}$ as in the GTVF~\cite{zhang_hu_adams17} in the computation of
gradient of velocity. The SPH discretization of the gradient of velocity is
given as,
\begin{equation}
  \label{eq:sph-vel-grad}
  \nabla \ten{u}_a =
  - \sum_{b} \frac{m_b}{\rho_{b}} (\ten{u}_{a} - \ten{u}_{b}) \otimes (\nabla_{a} W_{ab}),
\end{equation}
where $\otimes$ is the outer product.

The SPH discretization of the modified Jaumann stress rate
\cref{eq:modified-jaumann-stress-rate} is given as,
\begin{multline}
  \label{eq:sph-modified-jaumann-stress}
  \frac{\tilde{d}\teng{\sigma}^{'}_{a}}{dt} = 2G (\dot{\teng{\epsilon}}_{a} -
  \frac{1}{3} \dot{\teng{\epsilon}}_{a} \ten{I}) + \teng{\sigma}^{'}_{a}
  \teng{\Omega}_{a}^{T} +
  \teng{\Omega}_{a} \teng{\sigma}^{'}_{a} + \\
  + \sum_{b} \; \frac{m_b}{\rho_{b}} \; (\teng{\sigma}^{'} \otimes (\tilde{\ten{u}} -
  \ten{u}))_{ab} \; \cdot \nabla_{a} W_{ab}
  + \teng{\sigma}^{'}_{a} \sum_{b} \; \frac{m_b}{\rho_{b}} \;
  (\tilde{\ten{u}} - \ten{u})_{ab} \; \cdot \nabla_{a} W_{ab}.
\end{multline}


\subsection{Transport velocity computation of the mediums}

The particles in the current scheme are moved with the transport velocity,
\begin{equation}
  \label{eq:transport_velocity_position_derivative}
  \frac{d\ten{r}_a}{dt} = \ten{\tilde{u}}_a.
\end{equation}
%
The transport velocity is updated using,
\begin{equation}
  \label{eq:transport_velocity}
  \ten{\tilde{u}}_a(t + \Delta t) =\ten{u}_a(t) + \Delta t \; \frac{\tilde{d} \ten{u}_a}{dt} +
  \bigg(\frac{d \ten{u}_{a}}{dt}\bigg)_{\text{c}} \Delta t
\end{equation}

Where $\big(\frac{d \ten{u}_a}{dt}\big)_{\text{c}}$ is the homogenization
acceleration which ensures that the particle positions are homogeneous. In the
current work we have explored two kinds of homogenization accelerations, one
is a displacement based technique due to \citet{sun2017deltaplus}, and the
other is the iterative particle shifting technique due to
\citet{huang_kernel_2019}. These are discussed in the following.


\subsubsection{Transport velocity computation of fluid medium}
\label{sec:fluid_pst}

In \citet{sun2017deltaplus}, the particle shifting technique was implemented as
a particle displacement ($\delta \ten{r}$). This was modified in
\citet{sun_consistent_2019} to be computed as a change to the velocity. In the
present work we modify this to be treated as an acceleration to the particle
in order to unify the treatment of different PST methods.

Firstly, the velocity deviation based equation is given as,
\begin{equation}
  \label{eq:sun2019_pst}
  \delta \ten{u}_a = - \text{Ma} \; (2h) c_0 \sum_b \bigg[
  1 + R \bigg( \frac{W_{ab}}{W(\Delta x)} \bigg)^n  \bigg] \nabla_a W_{ab} V_b,
\end{equation}
%
it is modified to force based as,
\begin{equation}
  \label{eq:sun2019_pst}
  \bigg(\frac{d \ten{u}_a}{dt}\bigg)_{\text{c}} = - \frac{\text{Ma} \;
    (2h) c_0}{\Delta t} \sum_b \bigg[1 + R \bigg( \frac{W_{ab}}{W(\Delta x)} \bigg)^n
  \bigg] \nabla_a W_{ab} V_b,
\end{equation}
where $R$ is an adjustment factor to handle the tensile instability, and
$\text{Ma}$ is the mach number of the flow. $V_b$ is the volume of the
$b$\textsuperscript{th} particle. The acceleration is changed to account for
particles that are on the free surface. We use $R = 0.2$ and $n = 4$ as
suggested by \citet{sun_consistent_2019}.


\subsubsection{Transport velocity computation of solid medium}
\label{sec:elastic_solid_pst}

In \citet{sun2017deltaplus}, the particle shifting technique was implemented as
a particle displacement ($\delta \ten{r}$). This was modified in
\citet{sun_consistent_2019} to be computed as a change to the velocity. In the
present work we modify this to be treated as an acceleration to the particle
in order to unify the treatment of different PST methods.

Firstly, the velocity deviation based equation is given as,
\begin{equation}
  \label{eq:sun2019_pst}
  \delta \ten{u}_a = - \text{Ma} \; (2h) c_0 \sum_b \bigg[
  1 + R \bigg( \frac{W_{ab}}{W(\Delta x)} \bigg)^n  \bigg] \nabla_a W_{ab} V_b,
\end{equation}
%
it is modified to force based as,
\begin{equation}
  \label{eq:sun2019_pst}
  \bigg(\frac{d \ten{u}_a}{dt}\bigg)_{\text{c}} = - \frac{\text{Ma} \;
    (2h) c_0}{\Delta t} \sum_b \bigg[1 + R \bigg( \frac{W_{ab}}{W(\Delta x)} \bigg)^n
  \bigg] \nabla_a W_{ab} V_b,
\end{equation}
where $R$ is an adjustment factor to handle the tensile instability, and
$\text{Ma}$ is the mach number of the flow. $V_b$ is the volume of the
$b$\textsuperscript{th} particle. The acceleration is changed to account for
particles that are on the free surface. We use $R = 0.2$ and $n = 4$ as
suggested by \citet{sun_consistent_2019}.


\section{Fluid structure interaction}

\subsection{Coupling between fluid and solid}
\label{subsec:fsi-coupling}

Coupling is handled straight forwardly in SPH. While modelling the fluid phase
and treating the fluid-structure interactions, the structure particles are
assumed to be boundary particles. And from the boundary handling given in
Adami [14], we compute the pressure of the boundary particles from the
extrapolated equation (12) and correspondingly set its density using equation
(13). Please note that the pressure we set here are only pertaining to the fsi
force and does not correspond to the real pressure or density of the structure
particles.

The force acting on the fluid particles is now computed from the pressure set
using the adami boundary conditions and density set consequently. We mark
particles comprising the solid as $a$ and fluid as $i$, and the force acting
on the fluid particle $i$ is given as

\begin{equation}
  f_i^s = -m_i \sum_{a} m_a \bigg(\frac{p_i}{\rho_{i}^2} +
  \frac{p_a}{\rho_{a}^2} + \Pi_{ia} \bigg) \nabla_{i} W(x_{ia})
\end{equation}

Force on the structure particle due to the fluid particle computed using
Newton's third law. Since each structure particle experiences equal and
opposite force as experienced by the fluid particle, the expression for the
force on structure due to a fluid particle is

\begin{equation}
  f_a^F = -m_i \sum_{a} m_a \bigg(\frac{p_a}{\rho_{a}^2} +
  \frac{p_i}{\rho_{i}^2} + \Pi_{ai} \bigg) \nabla_{a} W(x_{ai})
\end{equation}


One more way of coupling is by using \citet{khayyer2018enhanced} formulation.



% \subsection{Total energy of the structure}
% \label{sec:total-energy-struct}

% \citet{khayyer2018enhanced} had an extra section where he discusses about the
% energy conservation properties of the proposed FSI scheme. He discusses
% different energies in structure, kinetic energy and strain energy.


\section{Summary equations of the full model to be solved}

\subsection{Summary equations fluid model}

\begin{equation}
  \label{eq:sph-discretization-continuity}
  \frac{\tilde{d}\rho_a}{dt} = \sum_{b} \; \frac{m_b}{\rho_{b}} \; (
  \rho_{a} \; \tilde{\ten{u}}_{ab} \; + \;
  (\rho \; (\tilde{\ten{u}} \; - \;
  \ten{u}))_{ab}) \; \cdot \nabla_{a} W_{ab},
\end{equation}

\begin{multline}
  \label{eq:sph-discretization-edac}
  \frac{\tilde{d}p_a}{dt} = \sum_{b} \; \frac{m_b}{\rho_{b}} \; \bigg(
  (p_{a} - \rho_{a} c_{s}^2) \; \ten{u}_{ab} \; + \;
  p_{a} \; \tilde{\ten{u}}_{ab} \; - \;
  (p \; (\tilde{\ten{u}} - \ten{u}))_{ab} \; + \; \\
  4 \; \nu_{edac}
  \frac{p_a - p_b}{(\rho_a + \rho_b) (r^2_{ab} + 0.01 h_{ab}^{2})} \ten{r}_{ab}
  \bigg) \; \cdot \nabla_{a} W_{ab}.
\end{multline}
%

\begin{multline}
  \label{eq:sph-momentum-fluid}
  \frac{\tilde{d}\ten{u}_{a}}{dt} = - \sum_{b} m_b \bigg[
  \bigg(\frac{p_a}{\rho_a^2} + \frac{p_b}{\rho_b^2}\bigg) \ten{I} -
  \bigg(\frac{\ten{A}_a}{\rho_a^2} + \frac{\ten{A}_b}{\rho_b^2} + \Pi_{ab}
  \ten{I} \bigg) \bigg]
  \cdot \nabla_{a} W_{ab} \\
  + \ten{u}_{a} \sum_{b} \frac{m_b}{\rho_{b}} \; \tilde{\ten{u}}_{ab} \cdot
  \nabla_{a} W_{ab} + \sum_{b} m_b \frac{4 \eta \nabla W_{ab}\cdot
    \ten{r}_{ab}}{(\rho_a + \rho_b) (r_{ab}^2 + 0.01 h_{ab}^2)} \ten{u}_{ab} +
  \ten{g}_{a},
\end{multline}


\begin{equation}
  \label{eq:transport_velocity}
  \ten{\tilde{u}}_a(t + \Delta t) =\ten{u}_a(t) + \Delta t \; \frac{\tilde{d} \ten{u}_a}{dt} +
  \bigg(\frac{d \ten{u}_{a}}{dt}\bigg)_{\text{c}} \Delta t
\end{equation}


\begin{equation}
  \label{eq:gtvf_pst}
  \bigg(\frac{d \ten{u}_a}{dt}\bigg)_{\text{c}} = p_a^0 \;
  \sum_b m_b \frac{1}{\rho_a^2} \nabla_a \widetilde{W}_{ab} ,
\end{equation}


\subsection{Summary equations solid model}


\begin{equation}
  \label{eq:sph-discretization-continuity}
  \frac{\tilde{d}\rho_a}{dt} = \sum_{b} \; \frac{m_b}{\rho_{b}} \; (
  \rho_{a} \; \tilde{\ten{u}}_{ab} \; + \;
  (\rho \; (\tilde{\ten{u}} \; - \;
  \ten{u}))_{ab}) \; \cdot \nabla_{a} W_{ab},
\end{equation}


\begin{equation}
  \label{eq:sph-momentum-solid}
  \frac{\tilde{d}\ten{u}_{a}}{dt} = - \sum_{b} m_b \bigg[
  \bigg(\frac{p_a}{\rho_a^2} + \frac{p_b}{\rho_b^2}\bigg) \ten{I} -
  \bigg(\frac{\teng{\sigma}^{'}_{a}}{\rho_a^2} +
  \frac{\teng{\sigma}^{'}_{b}}{\rho_b^2} + \Pi_{ab} \ten{I} \bigg) \bigg]  \cdot \nabla_{a} W_{ab} +
  \ten{g}_{a},
\end{equation}

\begin{equation}
  \label{eq:sph-modified-jaumann-stress}
  \frac{\tilde{d}\teng{\sigma}^{'}_{a}}{dt} = 2G (\dot{\teng{\epsilon}}_{a} -
  \frac{1}{3} \dot{\teng{\epsilon}}_{a} \ten{I}) + \teng{\sigma}^{'}_{a}
  \teng{\Omega}_{a}^{T} +
  \teng{\Omega}_{a} \teng{\sigma}^{'}_{a}
\end{equation}

\begin{equation}
  \label{eq:strain-tensor}
  \dot{\epsilon}_{ij} = \frac{1}{2} \bigg(\frac{\partial u_i}{\partial x_j} +
  \frac{\partial u_j}{\partial x_i} \bigg),
\end{equation}
and $\Omega_{ij}$ is the rotation tensor,
\begin{equation}
  \label{eq:rotational-tensor}
  \Omega_{ij} = \frac{1}{2} \bigg(\frac{\partial u_i}{\partial x_j} -
  \frac{\partial u_j}{\partial x_i} \bigg).
\end{equation}


\begin{equation}
  \label{eq:sph-vel-grad}
  \nabla \ten{u}_a =
  - \sum_{b} \frac{m_b}{\rho_{b}} (\ten{u}_{a} - \ten{u}_{b}) \otimes (\nabla_{a} W_{ab}),
\end{equation}


\begin{equation}
  \label{eq:transport_velocity}
  \ten{\tilde{u}}_a(t + \Delta t) =\ten{u}_a(t) + \Delta t \; \frac{\tilde{d} \ten{u}_a}{dt} +
  \bigg(\frac{d \ten{u}_{a}}{dt}\bigg)_{\text{c}} \Delta t
\end{equation}


\begin{equation}
  \label{eq:sun2019_pst}
  \bigg(\frac{d \ten{u}_a}{dt}\bigg)_{\text{c}} = - \frac{\text{Ma} \;
    (2h) c_0}{\Delta t} \sum_b \bigg[1 + R \bigg( \frac{W_{ab}}{W(\Delta x)} \bigg)^n
  \bigg] \nabla_a W_{ab} V_b,
\end{equation}


\section{Results and discussion}
\label{sec:results}

We can get many simple to advanced benchmarks in this area by looking at
applied ocean engineering, coastal engineering, journal of fluids and
structures, marine structures and ocean engineering papers.


% =========================================
% =========================================
% start
% =========================================
% =========================================
\subsection{Hwang 2014 static cantilever beam}
\label{sec:hwang-2014-static-cantilever-beam}
Currently this is a sub problem of elastic dam break. In that problem we are
simulating gate with out considering the gravity. By considering the gravity
we get float division error. It will be interesting to see why such thing is
happening only in the case with gravity and solving such is fun.


\begin{table}[!ht]
  \centering
  \begin{tabular}[!ht]{ll}
    % \toprule
    Quantity & Values \\
    % \midrule
    $D$, Diameter & 2m \\
    $\rho_0$, reference density & 1000kg/m\textsuperscript{3} \\
    $c_s$ & 10m/s \\
    $D/\Delta x_{\max}$, lowest resolution & 4 \\
    $D/\Delta x_{\min} $, highest resolution & 160, 250, 500\\
    $C_r$ & 1.08 \\
    Reynolds number & 40, 550, 1000, 3000, and 9500 \\
    Time of simulation & 6 \\
    % \bottomrule
  \end{tabular}
  \caption{Parameters used for the flow past a circular cylinder problem.}%
  \label{tab:fpc-params}
\end{table}


\subsection{Dynamics response of an elastic plate}
\label{sec:elastic-plate}

In this example we validate the ETVF scheme in modeling an elastic structure.
This example is taken from \cite{Sun2019study}, who took it from Turek and
Hron, who simulated it using FEM.

An elastic plate is attached to a rigid cylinder. The material properties of
the elastic plate are as follows, the Young's modulus is $1.4\times10^6Pa$ and
a Poisson ratio of $0.4$ and the density is $1000 kg/m^3$. This plate is held
free in a gravity field whose value is $2m/s^2$.


Here we simulate a high speed elastic aluminum wedge impacting on undisturbed
water surface. This benchmark was dealt in
{A $\delta$ SPH-SPIM coupled method for fluid-structure interaction problems} and
{abbas khayyer} papers.
This is an attractive benchmark as it has an semi-analytical solution.

A fluid of length 6 m and a height of 2.2 m is settled inside a tank of length
6m and height 4m which gives rise to a pressure distribution of a hydrostatic
tank. Into this hydrostatic tank a wedge of length 0.6m and a thickness of
0.04m is been dropped with a velocity of 30 m/s. The wedge angle is 10
degrees. Please see figure \ref{fig:half-wedge-deformable} for visual clarity.

The initial particle spacing is set as 0.005m and the total number of
particles combining the wedge, tank and fluid is TOBEDONE. The density of the
fluid is taken as 1000 $kg/m^3$. The material properties of the wedge are as
follows: density of 2700 $Kg/m^3$, Young's modulus of 6.75 1e10 $N/m^2$ and a
Poisson ratio of 0.34. An artificial viscosity $alpha$ of 0.1 is used for
fluids and 1 for solid phase.

In figure \ref{wedge-entering-the-tank}, at different time stamps, we plot the
pressure fields of water and sigma 00 if the flexible wedge. (We need to give
some physical insight of what is expected and what is been observed, provided
our simulations are correct)



A total of three points are marked on the wedge for a qualitative validation.
In figure \ref{point-A-wedge-displacement}, we plot the displacement of point
A with time, and compared with Abbhas and and the semi analytical solution of
scolon 2004. Similarly at point B and point C we compare the pressure of the
current scheme with the one provided with Abbas and the semi analytical
solution. It can be seen that the current scheme is able to produce the
results with a reasonable accuracy.



\subsection{Ng 2020 Hydrostatic water column on an elastic plate}
\label{sec:hydrostatic-water-column-on-an-composite-elastic-plate}

In the first example, a hydrostatic water column on an elastic plate is
considered. The initial setup of the system with water pressure is shown in
figure, where an initial column of fluid under hydrostatic state is rested on an
elastic plate under gravity. The water column is $2$ m in height and $0.5$ m in
width, while the elastic plate is of $0.05$ m in thickness and $0.5$ m in
length, respectively. The elastic plate considered here is made of Aluminium,
whose material properties are: Young's modulus $E = 67.5$ GPa, Poisson's ratio
$\nu = 0.4$. According to the analytical solution \cite{sfsfd}, the deflection
magnitude at mid-span after the system's equilibrium state is
$d = -6.85 \times 10^{-5}$ m.



\begin{figure}[!htpb]
  \centering
  \includegraphics[width=0.4\textwidth]{{{figures/ng_2020_hydrostatic_water_column_on_elastic_plate/schematic_t_0.0}}}
  \caption{Hydrostatic water column on an elastic plate}
\label{fig:hs-water-on-plate}
\end{figure}

%

The simulation is ran for a total time of 0.4 s. For convergence study, three
different particle resolutions, respectively $d0 = 0.55 \times 10^{-2}$ m,
$0.35 \times 10^{-2}$ m, $0.25 \times 10^{-2}$ m, are adopted, which generates
10, 20 and 30 particles in along the width direction on the structure
respectively. The test case is ran for a period of 0.3 seconds with 1232233
particles for the highest resolution. The time step of the fluid and solid phase
are computed separately as per equation \eqref{eq:time_step}, but we use
substepping to update the fluid and solid phases as per section
\ref{section:substepping}.

%
\begin{figure}[!htpb]
  \centering
%
  \begin{subfigure}{0.49\textwidth}
    \centering
    \includegraphics[width=0.7\textwidth]{{{figures/ng_2020_hydrostatic_water_column_on_elastic_plate/snap_t_0.3}}}
  \end{subfigure}
%
  \begin{subfigure}{0.49\textwidth}
    \centering
    \includegraphics[width=0.3\textwidth]{{{figures/ng_2020_hydrostatic_water_column_on_elastic_plate/colorbar_t_0.0}}}
  \end{subfigure}
  \caption{Pressure and stress distribution in an hydrostatic tank example}
\label{fig:ng2020hsplate:t_0.3}
\end{figure}
%


%
\begin{figure}[!htpb]
  \centering
%
  \begin{subfigure}{0.8\textwidth}
    \centering
    \includegraphics[width=1\textwidth]{{{figures/ng_2020_hydrostatic_water_column_on_elastic_plate/y_amplitude}}}
  \end{subfigure}
  \caption{The mid-span deflection of the structure with time}
\label{fig:ng2020hsplate:deflection}
\end{figure}
%

\FloatBarrier%

% %
%   \begin{subfigure}{0.24\textwidth}
%     \centering
%     \includegraphics[width=0.7\textwidth]{{{figures/ng_2020_hydrostatic_water_column_on_elastic_plate/snap_t_0.2}}}
%     \subcaption{t = 4e-03 sec}\label{fig:rings:sun2019-nu-0.3975-2}
%   \end{subfigure}
% %
%   \begin{subfigure}{0.24\textwidth}
%     \centering
%     \includegraphics[width=0.7\textwidth]{{{figures/ng_2020_hydrostatic_water_column_on_elastic_plate/snap_t_0.2}}}
%     \subcaption{t = 7.3e-03 sec}\label{fig:rings:sun2019-nu-0.3975-3}
%   \end{subfigure}
% %
%   \begin{subfigure}{0.24\textwidth}
%     \centering
%     \includegraphics[width=0.7\textwidth]{{{figures/ng_2020_hydrostatic_water_column_on_elastic_plate/colorbar_t_0.0}}}
%     % \subcaption{t = 1.45e-02 sec}\label{fig:rings:sun2019-nu-0.3975-4}
%   \end{subfigure}
% % %
% %   \begin{subfigure}{0.3\textwidth}
% %     \centering
% %     \includegraphics[width=1.0\textwidth]{figures/rings/etvf_sun2019_poisson_ratio_0.3975/time5}
% %     \subcaption{t = 1.5e-02 sec}\label{fig:rings:sun2019-nu-0.3975-5}
% %   \end{subfigure}



\subsection{Water impact onto a forefront elastic plate}
\label{sec:water-impact-forefront}

In this example the elastic plate is placed at the end. Which gets impacted
due to the moving fluid. Further information can be found at section 3.2 of
\cite{liu2013numerical}, \citet{sun2019fully}, {A $\delta$ SPH-SPIM coupled
  method for fluid-structure interaction problems}.

\begin{figure}[!htpb]
  \centering
  \includegraphics[width=0.8\textwidth]{{{figures/sun_2019_dam_breaking_flow_impacting_an_elastic_plate/schematic_t_0.0}}}
  \caption{Dam breaking on to an elastic obstacle}
\label{fig:dambreak-onto-a-plate}
\end{figure}

%

\begin{figure}[!htpb]
  \centering
%
  \begin{subfigure}{0.49\textwidth}
    \centering
    \includegraphics[width=1.0\textwidth]{{{figures/sun_2019_dam_breaking_flow_impacting_an_elastic_plate/snap_t_0.0}}}
  \end{subfigure}
%
  \begin{subfigure}{0.49\textwidth}
    \centering
    \includegraphics[width=1.0\textwidth]{{{figures/sun_2019_dam_breaking_flow_impacting_an_elastic_plate/snap_t_0.2}}}
  \end{subfigure}

  \begin{subfigure}{0.49\textwidth}
    \centering
    \includegraphics[width=1.0\textwidth]{{{figures/sun_2019_dam_breaking_flow_impacting_an_elastic_plate/snap_t_0.4}}}
  \end{subfigure}
%
  \begin{subfigure}{0.49\textwidth}
    \centering
    \includegraphics[width=1.0\textwidth]{{{figures/sun_2019_dam_breaking_flow_impacting_an_elastic_plate/snap_t_0.5}}}
  \end{subfigure}


  \begin{subfigure}{0.4\textwidth}
    \centering
    \includegraphics[width=1.0\textwidth]{{{figures/sun_2019_dam_breaking_flow_impacting_an_elastic_plate/snap_t_0.6}}}
  \end{subfigure}
%
  \begin{subfigure}{0.4\textwidth}
    \centering
    \includegraphics[width=1.0\textwidth]{{{figures/sun_2019_dam_breaking_flow_impacting_an_elastic_plate/snap_t_0.7}}}
  \end{subfigure}
%
  \begin{subfigure}{0.09\textwidth}
    \centering
    \includegraphics[width=1.0\textwidth]{{{figures/sun_2019_dam_breaking_flow_impacting_an_elastic_plate/colorbar_t_0.0}}}
  \end{subfigure}


  \caption{Pressure and stress distribution in dam breaking onto a elastic
    obstacle case}
  \label{fig:sun2019dambreakobstable}
\end{figure}
%


\begin{figure}[!htpb]
  \centering
%
  \begin{subfigure}{0.8\textwidth}
    \centering
    \includegraphics[width=1\textwidth]{{{figures/sun_2019_dam_breaking_flow_impacting_an_elastic_plate/x_amplitude}}}
  \end{subfigure}
  \caption{The mid-span deflection of the structure with time}
\label{fig:ng2020hsplate:deflection}
\end{figure}
%
\FloatBarrier%

\subsection{Dam break with elastic gate}
\label{sec:dam-break-elastic-gate}

Taken from \citet{sun2019fully, ng2020coupled}.

A block of fluid, which is suddenly released under gravity hits an elastic
plate. This benchmark is simulated in many papers.

The material properties of the elastic gate are as follows, and Young's
modulus of $0.01 GPa$ with a Poisson ratio of $0.35$ and a density of
$2500 kg/m^3$.

The length of the fluid block is taken as $0.1m$, and the height is $0.3m$,
and the length of the elastic gate is $0.03m$ with a height of $0.05m$ and is
placed at a distance of $0.1m$ from the end of the fluid block.


As the simulation starts the fluid starts flowing under the influence of
gravity and after some time, it hits the elastic gate. Due to the influence of
the fluid the elastic gate deforms and due to the elastic gate the fluid
starts raising. The dynamics of both mediums is influenced by one another.


We have compared the the tip of the elastic gate with time against the
experimental results produced by \cite{xxx}. As can be seen the results
produced by the current scheme are accurate and are able to reproduce the
experimental results.



\subsection{High speed impact of an elastic aluminum wedge on undisturbed
  water surface}
\label{sec:wedge-impact-on-water}

Here we simulate a high speed elastic aluminum wedge impacting on undisturbed
water surface. This benchmark was dealt in
{A $\delta$ SPH-SPIM coupled method for fluid-structure interaction problems} and
{abbas khayyer} papers.
This is an attractive benchmark as it has an semi-analytical solution.

A fluid of length 6 m and a height of 2.2 m is settled inside a tank of length
6m and height 4m which gives rise to a pressure distribution of a hydrostatic
tank. Into this hydrostatic tank a wedge of length 0.6m and a thickness of
0.04m is been dropped with a velocity of 30 m/s. The wedge angle is 10
degrees. Please see figure \ref{fig:half-wedge-deformable} for visual clarity.

The initial particle spacing is set as 0.005m and the total number of
particles combining the wedge, tank and fluid is TOBEDONE. The density of the
fluid is taken as 1000 $kg/m^3$. The material properties of the wedge are as
follows: density of 2700 $Kg/m^3$, Young's modulus of 6.75 1e10 $N/m^2$ and a
Poisson ratio of 0.34. An artificial viscosity $alpha$ of 0.1 is used for
fluids and 1 for solid phase.

In figure \ref{wedge-entering-the-tank}, at different time stamps, we plot the
pressure fields of water and sigma 00 if the flexible wedge. (We need to give
some physical insight of what is expected and what is been observed, provided
our simulations are correct)

A total of three points are marked on the wedge for a qualitative validation.
In figure \ref{point-A-wedge-displacement}, we plot the displacement of point
A with time, and compared with Abbhas and and the semi analytical solution of
scolon 2004. Similarly at point B and point C we compare the pressure of the
current scheme with the one provided with Abbas and the semi analytical
solution. It can be seen that the current scheme is able to produce the
results with a reasonable accuracy.



\section{Conclusions}
\label{sec:conclusions}


\section*{References}
\bibliographystyle{model6-num-names}
\bibliography{references}



% This has two simulations, one taken from \cite{ng2020coupled} and \cite{khayyer2021coupled}.
% \begin{table}[!ht]
%   \centering
%   \begin{tabular}[!ht]{ll}
%     % \toprule
%     Quantity & Values \\
%     % \midrule
%     $D$, Diameter & 2m \\
%     $\rho_0$, reference density & 1000kg/m\textsuperscript{3} \\
%     $c_s$ & 10m/s \\
%     $D/\Delta x_{\max}$, lowest resolution & 4 \\
%     $D/\Delta x_{\min} $, highest resolution & 160, 250, 500\\
%     $C_r$ & 1.08 \\
%     Reynolds number & 40, 550, 1000, 3000, and 9500 \\
%     Time of simulation & 6 \\
%     % \bottomrule
%   \end{tabular}
%   \caption{Parameters used for the flow past a circular cylinder problem.}%
%   \label{tab:fpc-params}
% \end{table}
% %
%

\end{document}

%%% Local Variables:
%%% mode: latex
%%% TeX-master: t
%%% End:
