\documentclass[preprint,12pt]{elsarticle}

\usepackage{hyperref}
\usepackage{graphicx}
\usepackage{subcaption}
\usepackage{amssymb}
\usepackage{amsmath}
\usepackage{multirow}
\usepackage[utf8]{inputenc}
\usepackage{cleveref}

% For the TODOs
\usepackage{xcolor}
\usepackage{xargs}
\usepackage[colorinlistoftodos,textsize=footnotesize]{todonotes}
\newcommand{\todoin}{\todo[inline]}
% from here: https://tex.stackexchange.com/questions/9796/how-to-add-todo-notes
\newcommandx{\unsure}[2][1=]{\todo[linecolor=red,backgroundcolor=red!25,bordercolor=red,#1]{#2}}
\newcommandx{\change}[2][1=]{\todo[linecolor=blue,backgroundcolor=blue!25,bordercolor=blue,#1]{#2}}
\newcommandx{\info}[2][1=]{\todo[linecolor=OliveGreen,backgroundcolor=OliveGreen!25,bordercolor=OliveGreen,#1]{#2}}

%Boldtype for greek symbols
\newcommand{\teng}[1]{\ensuremath{\boldsymbol{#1}}}
\newcommand{\ten}[1]{\ensuremath{\mathbf{#1}}}

\usepackage{lineno}

\journal{}

\begin{document}

\begin{frontmatter}

  \title{Fluid structure interaction with ETVF}
  \author[IITB]{A Dinesh\corref{cor1}}
  \ead{adepu.dinesh.a@gmail.com}
  \author[IITB]{Prabhu Ramachandran}
  \ead{prabhu@aero.iitb.ac.in}
  \address[IITB]{Department of Aerospace Engineering, Indian Institute of
    Technology Bombay, Powai, Mumbai 400076}

\cortext[cor1]{Corresponding author}

\begin{abstract}
\end{abstract}

\begin{keyword}
%% keywords here, in the form: keyword \sep keyword
{XXX}, {XXX}, {XXX}

%% MSC codes here, in the form: \MSC code \sep code
%% or \MSC[2008] code \sep code (2000 is the default)

\end{keyword}

\end{frontmatter}

% \linenumbers



\section{Introduction}
\label{sec:intro}





\section{Fluid structure coupling}
\label{sec:fsi-coupling}

Coupling is handled straight forwardly in SPH. While modelling the fluid phase
and treating the fluid-structure interactions, the structure particles are
assumed to be boundary particles. And from the boundary handling given in
Adami [14], we compute the pressure of the boundary particles from the
extrapolated equation (12) and correspondingly set its density using equation
(13). Please note that the pressure we set here are only pertaining to the fsi
force and does not correspond to the real pressure or density of the structure
particles.

The force acting on the fluid particles is now computed from the pressure set
using the adami boundary conditions and density set consequently. We mark
particles comprising the solid as `a` and fluid as `i`, and the force acting
on the fluid particle `i` is given as

\begin{equation}
  f_i^s = -m_i \Sum_{a} m_a \bigg(\frac{p_i}{\rho_{i}^2} +
  \frac{p_a}{\rho_{a}^2} + \Pi_{ia} \bigg) \nabla_{i} W(x_{ia})
\end{equation}

Force on the structure particle due to the fluid particle computed using
Newton's third law. Since each structure particle experiences equal and
opposite force as experienced by the fluid particle, the expression for the
force on structure due to a fluid particle is

\begin{equation}
  f_a^F = -m_i \Sum_{a} m_a \bigg(\frac{p_a}{\rho_{a}^2} +
  \frac{p_i}{\rho_{i}^2} + \Pi_{ai} \bigg) \nabla_{a} W(x_{ai})
\end{equation}




\section{Results and discussion}
\label{sec:results}

We can get many simple to advanced benchmarks in this area by looking at
applied ocean engineering, coastal engineering, journal of fluids and
structures, marine structures and ocean engineering papers.


% =========================================
% =========================================
% start
% =========================================

\subsection{Hydrostatic water column on an elastic plate}
\label{sec:results:hstank-elastic-plate}


In this problem we check the accuracy and stability-preserving property of the
proposed scheme in handling the fluid structure interaction. This problem has
been studied in Li et al. (2015), Fourey et al. (2017), Khayyer et
al. (2018a), Hu et al. (2019), (2021), (2021) papers. This benchmark has an
analytical solution derived in Li et al. (2015).

We can see the problem description in figure XXX. The water column height is
2m and has a width of 0.5m. An aluminum plate of thickness 0.05m is placed at
the bottom of the tank.The material properties of the aluminum are as follows,
a density of 2700 kg/m^3, an Young’s modulus of $E = 6.75 \times 10^{10} Pa$,
and the Poisson’s ratio $\nu = 0.34$. The plate is clamped on both the
sides. Once the simulation begins, a sudden hydrostatic pressure force is
applied on the aluminum plate. With time the plate oscillates and comes to
rest. We compare the static displacement of the plate with the analytical
solution.

In the current simulation a total of two particle spacing is used. One with
$\delta x = 0.04 , 0.02m and 0.01 m$ and use quintic spline as a kernel. We
use 1e-6, 1e-7 for timestepping of fluid, solid phases respectively. All the
simulations are run for a total time of $t_f = 1e-4s$. In figures XXX, XXX,
XXX we show the final positions of the particles with different particle
resolutions. The fluid contours are pressure and the structure contours is the
y-displacement. In figure XXX we plot the midpoint deflection with time and
compare it with the analytical solution provided by khayyer . It can be seen
that the results produced by the current scheme are accurate and validates the
scheme for further use in FSI problems. We have run a convergence analysis
with a reducing spacing and as can be seen in the figure, that as the particle
spacing is reduced the error is reduced.

Also analyse any other extra plots you are planning to put.

% =========================================
% end
% =========================================



\subsection{Elastic bar hanging under gravity}
\label{sec:elastic-bar-hanging}


Currently this is a sub problem of elastic dam break. In that problem we are
simulating gate with out considering the gravity. By considering the gravity
we get float division error. It will be interesting to see why such thing is
happening only in the case with gravity and solving such is fun.


\subsection{Dam break with elastic gate}
\label{sec:dam-break-elastic-gate}

A block of fluid, which is suddenly released under gravity hits an elastic
plate. This benchmark is simulated in many papers.

The material properties of the elastic gate are as follows, and Young's
modulus of $0.01 GPa$ with a Poisson ratio of $0.35$ and a density of
$2500 kg/m^3$.

The length of the fluid block is taken as $0.1m$, and the height is $0.3m$,
and the length of the elastic gate is $0.03m$ with a height of $0.05m$ and is
placed at a distance of $0.1m$ from the end of the fluid block.


As the simulation starts the fluid starts flowing under the influence of
gravity and after some time, it hits the elastic gate. Due to the influence of
the fluid the elastic gate deforms and due to the elastic gate the fluid
starts raising. The dynamics of both mediums is influenced by one another.


We have compared the the tip of the elastic gate with time against the
experimental results produced by \cite{xxx}. As can be seen the results
produced by the current scheme are accurate and are able to reproduce the
experimental results.





\subsection{Water impact onto a forefront elastic plate}
\label{sec:water-impact-forefront}

In this example the elastic plate is placed at the end. Which gets
impacted due to the moving fluid. Further information can be found at
section 3.2 of \cite{liu2013numerical}






\section{Conclusions}
\label{sec:conclusions}


\section*{References}
\bibliographystyle{model6-num-names}
\bibliography{references}


\end{document}

%%% Local Variables:
%%% mode: latex
%%% TeX-master: "paper"
%%% fill-column: 78
%%% End:
