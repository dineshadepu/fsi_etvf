\documentclass[preprint,12pt]{elsarticle}

\usepackage{hyperref}
\usepackage{graphicx}
\usepackage{subcaption}
\usepackage{amssymb}
\usepackage{amsmath}
\usepackage{multirow}
\usepackage[utf8]{inputenc}
\usepackage{cleveref}

% For the TODOs
\usepackage{xcolor}
\usepackage{xargs}
\usepackage[colorinlistoftodos,textsize=footnotesize]{todonotes}
\newcommand{\todoin}{\todo[inline]}
% from here: https://tex.stackexchange.com/questions/9796/how-to-add-todo-notes
\newcommandx{\unsure}[2][1=]{\todo[linecolor=red,backgroundcolor=red!25,bordercolor=red,#1]{#2}}
\newcommandx{\change}[2][1=]{\todo[linecolor=blue,backgroundcolor=blue!25,bordercolor=blue,#1]{#2}}
\newcommandx{\info}[2][1=]{\todo[linecolor=OliveGreen,backgroundcolor=OliveGreen!25,bordercolor=OliveGreen,#1]{#2}}

%Boldtype for greek symbols
\newcommand{\teng}[1]{\ensuremath{\boldsymbol{#1}}}
\newcommand{\ten}[1]{\ensuremath{\mathbf{#1}}}

\usepackage{lineno}

\journal{}

\begin{document}

\begin{frontmatter}

  \title{Fluid structure interaction and rigid fluid coupling in SPH}
  \author[IITB]{A Dinesh\corref{cor1}}
  \ead{adepu.dinesh.a@gmail.com}
  \author[IITB]{Prabhu Ramachandran}
  \ead{prabhu@aero.iitb.ac.in}
  \address[IITB]{Department of Aerospace Engineering, Indian Institute of
    Technology Bombay, Powai, Mumbai 400076}

\cortext[cor1]{Corresponding author}

\begin{abstract}
\end{abstract}

\begin{keyword}
%% keywords here, in the form: keyword \sep keyword
{XXX}, {XXX}, {XXX}

%% MSC codes here, in the form: \MSC code \sep code
%% or \MSC[2008] code \sep code (2000 is the default)

\end{keyword}

\end{frontmatter}

% \linenumbers



\section{Introduction}
\label{sec:intro}


\section{The SPH method}
\label{sec:sph}


\section{2D Rigid body dynamics}
\label{sec:rb-2d}


\subsection{Center of mass, Linear momentum of a rigid body}
\label{sec:center-mass}

Let $N$ be an inertial frame in which a body $B$ is situated and is
discretized into finite number of points. The center of mass is defined as \cite{rao2006dynamics}


\begin{equation}
  \label{eq:center-of-mass}
  \ten{r}_{cm} = \frac{\sum_i m_i \; \ten{r}_i}{M}
\end{equation}

\noindent where $\ten{r}$ is the position vector of the point $i$ with mass
$m_i$ in inertial frame $N$ and $M$ is the total mass of the rigid body under
consideration.


The linear momentum of the rigid body in the inertial frame is


\begin{equation}
  \label{eq:lin-mom-of-rigid-body}
  \ten{G} = \sum_i m_i \; \ten{v}_i
\end{equation}


\begin{equation}
  \label{eq:center-of-mass-velocity}
  \ten{v}_{cm} = \frac{\sum_i m_i \; \ten{v}_i}{M}
\end{equation}

\noindent where $\ten{v}$ is the velocity vector of the point mass $i$ as
observed in the inertial frame. Similarly the acceleration of the center of
mass is


\begin{equation}
  \label{eq:acceleration-of-rigid-body}
  \ten{a}_{cm} = \frac{\sum_i m_i \; \ten{a}_i}{M}
\end{equation}



\section{Rigid body interaction}
\label{sec:rigid-body-inter}

As in




\section{Results and discussion}
\label{sec:results}

\subsection{Cylinder column collapse under the action of gravity}
\label{sec:cylinder-column-collapse}


This benchmark is to test the contact law between the rigid bodies.


\subsection{Regular waves interaction with a freely floating box}
\label{sec:cylinder-column-collapse}

This benchmark is used to test the interaction between fluid and rigid bodies
aka rigid-fluid coupling. Please refer section 3.1 of \cite{dominguez2019sph}
for further details.

\subsection{Symmetric water entry of a light wedge (2D)}
\label{sec:Symmetric-water-entry}

This benchmark is also to test the interaction between fluid and rigid bodies
aka rigid-fluid coupling. Please refer section 3.4 of
\cite{amicarelli2015smoothed} for further details.


\subsection{Asymmetric water entry of a light wedge (2D)}
\label{sec:Asymmetric-water-entry}

This benchmark is also to test the interaction between fluid and rigid bodies
aka rigid-fluid coupling. Please refer section 3.4 of
\cite{amicarelli2015smoothed} for further details.


\subsection{Dam break with body transport (3D)}
\label{sec:dam-break-body}

This benchmark is to test both rigid fluid coupling as well as the rigid body
interaction. See section 3.6 of \cite{amicarelli2015smoothed} for further
details.


\subsection{Demonstrative test case: multiple body transport (3D)}
\label{sec:demonstrative-test-case}

This benchmark is to test both rigid fluid coupling as well as the rigid body
interaction. See section 3.7 of \cite{amicarelli2015smoothed} for further
details.


\subsection{Elastic bar hanging under gravity}
\label{sec:elastic-bar-hanging}


Currently this is a sub problem of elastic dam break. In that problem we are
simulating gate with out considering the gravity. By considering the gravity
we get float division error. It will be interesting to see why such thing is
happening only in the case with gravity and solving such is fun.


\subsection{Dam break with elastic gate}
\label{sec:dam-break-elastic-gate}

This is to test fluid structure interaction.


\subsection{Water impact onto a forefront elastic plate}
\label{sec:water-impact-forefront}

In this example the elastic plate is placed at the end. Which gets
impacted due to the moving fluid. Further information can be found at
section 3.2 of \cite{liu2013numerical}


\section{Conclusions}
\label{sec:conclusions}


\section*{References}
\bibliographystyle{model6-num-names}
\bibliography{references}


\end{document}

%%% Local Variables:
%%% mode: latex
%%% TeX-master: "paper"
%%% fill-column: 78
%%% End: